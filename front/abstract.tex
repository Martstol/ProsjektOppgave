\section*{Abstract}

In this project the wind simulator used in the snow simulator developed at
NTNU's HPC-lab will be implemented with the PETSc library. The goal of this
implementation is to use the GPU implementation of PETSc's solvers and to
compare the performance of these solvers with the previously implemented wind
simulation solvers. The performance of PETSc's GPU solvers will also be compared
with the performance of the solvers implemented in PETSc for multi-core CPU
systems to see if there is any gain from using the GPU implementations.

GPGPU programming is a recently developed area of high performance computing that
offers cost and energy efficient computing power by using many smaller and less
complex cores. Today GPGPU programming is mainly done using low level programming
APIs like CUDA and OpenCL. These APIs are considered to be challenging to
program with and may lead to complex software design.

PETSc is a library for high performance computing that implements matrix and vector
operations and a large selection of numerical solvers. PETSc is built to support
parallel programming with MPI and pthreads. In recent years PETSc has added
support for using the GPU with both CUDA and OpenCL. This enables usage of the
computational power of GPGPU programming for high performance
computing without having to program with low level APIs like CUDA and OpenCL.
The results show very good performance gains for using PETSc's GPU implementation.
