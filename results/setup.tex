\section{Setup}

This section will describe the hardware of the workstation used when testing the
implementation of the wind simulator and the software used for the
implementation.

\subsection{Workstation}

The implementation in this project was only tested on one workstation. The
hardware and software used in this workstation during testing can be found in
table \ref{table:test_pc}.

\begin{table}[h]
	\begin{center}
	\bgroup
	\def\arraystretch{1.2}
	\begin{tabular}{|l|l|}
		\hline
		\multicolumn{2}{|c|}{\textbf{Workstation hardware}} \\ \hline
		CPU & i7-2600 @ 3.40 GHz \\ \hline
		GPU & Nvidia GeForce GTX 970 \\ \hline
		Memory & 8 GB DDR3 1333 MHz \\ \hline
		\multicolumn{2}{|c|}{\textbf{Workstation software}} \\ \hline
		Operating system & Ubuntu 14.04 64-bit \\ \hline
		Nvidia driver & 343.36 \\ \hline
		CUDA toolkit & 6.5 and 5 \\ \hline
		OpenCL & 1.1 \\ \hline
		MPI & MPICH 3.1.3 \\ \hline
		g++ & 4.4.1 \\ \hline
		gcc & 4.4.1 \\ \hline
		PETSc & development version \\ \hline
	\end{tabular}
	\egroup
	\end{center}
	\caption{Specifications of the workstation used for testing the implementation.}
	\label{table:test_pc}
\end{table}

\subsection{PETSc}

PETSc was acquired from the development repository\footnote{\url{https://bitbucket.org/petsc/petsc}}
, this was recommended on the PETSc web page when using PETSc's GPU solvers.
The version of the compilers used when configuring PETSc can be found in table 
\ref{table:test_pc} as well as the OpenCL and CUDA version used. Older versions
of gcc and g++ was used because during this project PETSc only supported CUDA
version 5 which required an older version of gcc. An older version of g++ was
used for compatibility reasons when configuring PETSc.

\subsubsection{Configuration}

OpenCL with ViennaCL was chosen to test PETSc's GPU solvers. CUDA was not tested
because of issues with lacking backwards compatibility in the CUDA toolkit. While
PETSc only supports CUDA version 5, the OpenCL implementation in CUDA version 6.5
worked with PETSc. The following is the flags specified when configuring PETSc for
benchmarking:

\begin{description}[labelindent=1cm,font=\normalfont\space]
	\item[--with-cxx=g++]
	\item[--with-fc=0]
	\item[--with-mpi-dir=/usr/local/mpich-3.1.3]
	\item[--download-f2cblaslapack=yes]
	\item[--with-opencl=1]
	\item[--with-opencl-dir=/usr/local/cuda-6.5]
	\item[--with-viennacl=1]
	\item[--download-viennacl=yes]
	\item[--with-precision=single]
	\item[--with-errorchecking=0]
	\item[--with-debugging=0]
	\item[--CXXOPTFLAGS='O3']
	\item[--COPTFLAGS='O3']
\end{description}

For this configuration of PETSc all forms of debug information and error checking
was disabled to minimize the overhead. Single precision was also selected as
GPUs today have faster single precision floating point arithmetic, often by several
orders of magnitude and the rest of the snow simulator is already written with
single precision.
