\section{Setup}

This section will describe the hardware of the workstation used when testing the
implementation of the wind simulator and the software used for the
implementation.

\subsection{Workstation}

The implementation in this project was only tested on one workstation. The
hardware and software used in this workstation during testing can be found in
table \ref{table:test_pc}.

\begin{table}[h]
	\begin{center}
	\bgroup
	\def\arraystretch{1.2}
	\begin{tabular}{|l|l|}
		\hline
		\multicolumn{2}{|c|}{\textbf{Workstation hardware}} \\ \hline
		CPU & i7-2600 @ 3.40 GHz \\ \hline
		GPU & Nvidia GeForce GTX 970 \\ \hline
		Memory & 10 GB DDR3 1333 MHz \\ \hline
		Power supply & MIST AX1000 \\ \hline
		\multicolumn{2}{|c|}{\textbf{Workstation software}} \\ \hline
		Operating system & Ubuntu 14.04 64-bit \\ \hline
		Nvidia driver & 343.36 \\ \hline
		CUDA toolkit & 6.5 \\ \hline
		OpenCL & 1.1 \\ \hline
		g++ & 4.4.1 \\ \hline
		gcc & 4.4.1 \\ \hline
		PETSc & development version \\ \hline
	\end{tabular}
	\egroup
	\end{center}
	\caption{Specifications of the workstation used for testing the implementation.}
	\label{table:test_pc}
\end{table}

\subsection{PETSc}

PETSc was acquired from the git repository\footnote{\url{https://bitbucket.org/petsc/petsc}}
, this was recommended on the PETSc web page when utilizing PETSc's GPU solvers.
The version of the compilers used when configuring PETSc can be found in table 
\ref{table:test_pc} as well as the OpenCL and CUDA version used.

\subsubsection{Configuration}

OpenCL with ViennaCL was chosen to test PETSc's GPU solvers. CUDA was not tested
because of issues with lacking backwards compatability in the CUDA toolkit, this
is further discussed in chapter \ref{chap:discussion}. 

\lstset{language=bash}
\begin{lstlisting}
./configure --with-cc=gcc --with-cxx=g++ --with-fc=0
--with-debugging=0 --with-precision=single
COPTFLAGS='-O3' CXXOPTFLAGS='-O3' 
--download-f2cblaslapack --download-mpich
--with-cuda=1 --with-cuda-dir=/usr/local/cuda-6.5 
--with-cuda-arch=sm_50 --with-cusp=1 --download-cusp=yes
\end{lstlisting}
For this configuration of PETSc debugging was disabled and precision was set to 
single. This is because single precision is faster and some GPUs may not support 
double precision. The rest of the snow simulator also uses single precision.

The CUDA installation is located in the folder \emph{/usr/local/cuda-6.5}. The 
cuda architecture was set to \emph{sm\_50}. PETSc did not accept \emph{sm\_52}, 
which is the compute capability of the GTX 980. 

\subsection{Compilation}

