\section*{Finite Difference Method}

The Finite Difference Method (FDM) is a numerical method used to find an approximate 
solution to differential equations. It works by creating a discrete approximation 
of the derivative and using this approximation to write the differential equation 
as a system of linear equations.

There are different versions of the Finite Difference Method.
\begin{itemize}
	\item Forward difference, $\Delta_hf(x) = f(x+h) - f(x)$
	\item Backward difference, $\nabla_hf(x) = f(x) - f(x - h)$
	\item Central difference, $\delta_hf(x) = f(x + \frac{1}{2}h) - f(x - \frac{1}{2}h)$
\end{itemize}

It is easy to see the connection of the Forward difference and the definition of 
the derivative.

$f'(x) = \lim_{h \to 0} \frac{f(x+h) - f(x)}{h}$

We then give $h$ a fixed, non-zero, value instead of having $h$ approach 0. Therefore 
the forward difference divided by $h$ approximates the derivative when $h$ is small.

The second order derivative can be approximated using the following central difference formula:

$f''(x) \approx \frac{\delta_h^2f(x)}{h^2} = \frac{f(x+h) - 2f(x) + f(x-h)}{h^2}$

In three-dimensions the approximation looks like this:

$\frac{\partial^2 f(x, y, z)}{\partial^2 x} \approx \frac{f(x+h, y, z) - 2f(x, y, z) + f(x-h, y, z)}{h^2}$

$\frac{\partial^2 f(x, y, z)}{\partial^2 y} \approx \frac{f(x, y+h, z) - 2f(x, y, z) + f(x, y-h, z)}{h^2}$

$\frac{\partial^2 f(x, y, z)}{\partial^2 z} \approx \frac{f(x, y, z+h) - 2f(x, y, z) + f(x, y, z-h)}{h^2}$

$(\pTwo{x} + \pTwo{y} + \pTwo{z}) f (x, y, z) = \frac{\partial^2 f(x, y, z)}{\partial^2 x} 
+ \frac{\partial^2 f(x, y, z)}{\partial^2 y} + \frac{\partial^2 f(x, y, z)}{\partial^2 z}$

$(\pTwo{x} + \pTwo{y} + \pTwo{z}) f (x, y, z) \approx \frac{f(x+h, y, z) + f(x, y+h, z) + f(x, y, z+h) 
- 6f(x, y, z) + f(x-h, y, z) + f(x, y-h, z) + f(x, y, z+h)}{h^2}$

This is also known as the 7-point stencil.

\begin{figure}[h]
	\center
	\includegraphics[width=0.5\textwidth]{images/7_point_stencil}
	\caption{7-point stencil}
\end{figure}

%7 point stencil graphic

\subsection*{Discrete Poisson Equation}

Describe how the discrete Poisson Equation looks when using the second-order central 
difference approximation.
