\section*{Finite Difference Method}

The finite difference method (FDM) is a numerical method used to find an approximate 
solution to differential equations. It works by creating a discrete approximation 
of the derivative and using this approximation to write the differential equation 
as a system of linear equations.

There are different versions of the finite difference method.
\begin{itemize}
	\item Forward difference, $\Delta_hf(x) = f(x+h) - f(x)$
	\item Backward difference, $\nabla_hf(x) = f(x) - f(x - h)$
	\item Central difference, $\delta_hf(x) = f(x + \frac{1}{2}h) - f(x - \frac{1}{2}h)$
\end{itemize}

It is easy to see the connection of the forward difference and the definition of 
the derivative.

$$f'(x) = \lim_{h \to 0} \frac{f(x+h) - f(x)}{h}$$

We then give $h$ a fixed, non-zero, value instead of having $h$ approach 0. Therefore 
the forward difference divided by $h$ approximates the derivative when $h$ is small.

$$f'(x) \approx \frac{f(x+h) - f(x)}{h}$$

TODO: Write about the error in the approximation

The second order derivative can be approximated by applying a finite difference 
approximation to the finite difference approximation. The following is the result 
for the central difference:

$$f''(x) \approx \frac{\delta_h^2f(x)}{h^2} = \frac{f(x+h) - 2f(x) + f(x-h)}{h^2}$$

When applied to higher dimensions, it is necessary to choose a value for h in each dimension. However we can simplify the problem by choosing the same h. In three-dimensions the approximation looks like this:

$$(\pTwo{x} + \pTwo{y} + \pTwo{z}) f (x, y, z) = \frac{\partial^2 f(x, y, z)}{\partial^2 x} 
+ \frac{\partial^2 f(x, y, z)}{\partial^2 y} + \frac{\partial^2 f(x, y, z)}{\partial^2 z}$$

$$\frac{\partial^2 f(x, y, z)}{\partial^2 x} \approx \frac{f(x+h, y, z) - 2f(x, y, z) + f(x-h, y, z)}{h^2}$$

$$\frac{\partial^2 f(x, y, z)}{\partial^2 y} \approx \frac{f(x, y+h, z) - 2f(x, y, z) + f(x, y-h, z)}{h^2}$$

$$\frac{\partial^2 f(x, y, z)}{\partial^2 z} \approx \frac{f(x, y, z+h) - 2f(x, y, z) + f(x, y, z-h)}{h^2}$$

This is also known as the 7-point stencil as illustrated in figure \ref{7ps}.

\begin{figure}[ht]
	\center
	\includegraphics[width=0.5\textwidth]{images/7_point_stencil}
	\caption{7-point stencil}
	\label{7ps}
\end{figure}

%7 point stencil graphic

\subsection*{Discrete Poisson Equation}

When we want to apply the finite difference method to the Poisson equation with 
Dirichlet boundary conditions.
We have to sample the problem domain in a finite number of points. The distance
between the sample points, $h$, is the same as the fixed $h$ in the finite
differences method. Lets consider the two-dimensional Poisson equation $\nabla^2
u(x, y) = f(x, y)$. We use the central difference approximation to sample
$\nabla^2 u(x, y)$ at $n+1$ points in the $x$ direction and $n+1$ points in $y$
direction. We assume the dimensions of the domain goes from $0$ to $1$.

This way we have $(n-1)^2$ internal points and the remaining points are boundary
points that have their values specified by the boundary conditions. The total
number of unknowns are therefore $N = (n-1)^2$.

$$h = \frac{1}{n}$$
$$x_i = ih, ~~~~ i = 0, 1, \dots, n$$
$$y_j = jh, ~~~~ j = 0, 1, \dots, n$$

We then denote the approximation of the value of $u(x_i, y_j)$ by $u_{i,j}$ and 
$f(x_i, y_j)$ by $f_{i,j}$

$$ (\nabla^2 u)_{ij} \approx \frac{u_{i+1,j} + u_{i,j+1} - 4u_{i,j} + u_{i-1,j} + u_{i,j-1}}{h^2} = f_{i,j} $$
$$ u_{i+1,j} + u_{i,j+1} - 4u_{i,j} + u_{i-1,j} + u_{i,j-1} = h^2 f_{i,j} $$

This creates a system of linear equations, which we want to write on matrix form.

$$Ax = b$$

To create this linear system we have to create a global ordering for the
internal nodes to create the $x$ vector of unknowns. We use the natural
ordering, which means that we list the nodes along the $x$ direction first.

$$x_k = u_{i,j}, ~ k = (i-1) + (n-1) \cdot (j-1), ~ i, j = 1, 2, \dots, n-1$$

We can then assemble the matrix $A$.

$$
A = \begin{bmatrix}
 B & I & 0 & \cdots & 0 \\
 I & B & I &   & 0 \\
 \vdots &   & \ddots &   & \vdots \\
 \vdots &   &   & \ddots & \vdots \\
 0 & \vdots & \vdots & I & B 
\end{bmatrix}
~~
B = \begin{bmatrix}
-4 & 1 & 0 & \cdots & 0 \\
 1 &-4 & 1 &   & 0 \\
 \vdots &   & \ddots &   & \vdots \\
 \vdots &   &   & \ddots & \vdots \\
 0 & \vdots & \vdots & 1 &-4 
\end{bmatrix}
$$

$A$ is a $N \times N$ matrix, $B$ is a $(n-1) \times (n-1)$ matrix and $I$ is the identity matrix.

The vector $b$ is assembled from $h^2$, the function $f$, and the Dirichlet border conditions. 

$$ b_k = h^2 \cdot f_{i,j} - u_{?} $$

TODO: how is $b$ created
