\section*{Partial Differential Equations}

A Partial Differential Equation (PDE) is a differential equation with multiple 
unknown variables and their partial derivatives. PDEs can be used to model physical 
processes in multiple dimensions. 

\subsection*{Boundary Value Problem}

A boundary value problem is a differential equation and a set of conditions for the values on the boundary of the problem domain. 

There are several different types of boundary conditions. If the values on the boundary is given, then it is a Dirichlet boundary condition. If the derivative normal to the border is given then it is a Neumann boundary condition[SOURCE].

\subsection*{Computational Fluid Dynamics}

I'm not sure if I should include this chapter

\subsection*{Poisson Equation}

In general, the Poisson equation is defined as 

$$\Delta \varphi = f$$

where $\Delta$ is the Laplace operator and $\varphi$ and $f$ are real or complex 
functions. In three-dimensional Cartesian coordinates, the Poisson Equation can 
be written as 

$$(\pTwo{x} + \pTwo{y} + \pTwo{z}) \varphi (x, y, z) = f(x, y, z)$$

