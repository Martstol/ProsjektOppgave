\section{Computational Fluid Dynamics}

Fluid flows can be described by partial differential equations for the conservation 
of mass, momentum and energy. In computational fluid dynamics (CFD) the partial 
differential equations are replaced by a system of linear equations that can be 
solved by computers. 

\subsection{Navier-Stokes Equations}

Navier-Strokes equations are a set of partial differential equations which describes 
fluid flow\cite{fluidDynamicsIntro}. By assuming the fluid is incompressible, 
meaning that the volume does not change, the equations are simplified to the 
Navier-Strokes equations for incompressible flow\cite{originalSnowThesis}. 

\begin{equation} 
	\tag{momentum equation}
	\pOne{t}u  = -(u \cdot \nabla)u - \frac{1}{\rho}\nabla p + \nu \nabla^2 u + f
\end{equation}

\begin{equation}
	\tag{continuity equation}
	\nabla \cdot u = 0
\end{equation}

Here $u$ is the velocity vector field, $p$ is the pressure field, $\rho$ is the 
density of the fluid, $\nu$ is the kinematic viscosity and $f$ are external 
forces working on the fluid\cite{gpuGemsCh38}. Table \ref{table:vectorCalculus} 
explains the difference between the different uses of $\nabla$, also called the 
nabla operator\cite{gpuGemsCh38}.

\begin{table}[h]
	\begin{center}
	\bgroup
	\def\arraystretch{1.5}
	\begin{tabular}{ccc}
		\hline
		Operator & Definition & Example \\
		\hline
		Gradient & $ \nabla p $ & $\Big( \pOne{x}p, \pOne{y}p \Big)$ \\ \hline
		Divergence & $ \nabla \cdot u $ & $ \pOne{x}u_x + \pOne{y}u_y $ \\ \hline
		Laplacian & $ \nabla^2 p $ & $ \pTwo{x}p + \pTwo{y}p $ \\ \hline
	\end{tabular}
	\egroup
	\end{center}
	\caption{Usage of the nabla operator in CFD.}
	\label{table:vectorCalculus}
\end{table}

The gradient of the pressure field is a vector of partial derivatives of the 
scalar field, also called the gradient vector field. Divergence leads to a sum of 
the partial derivatives of each component of the vectors in a vector field. The 
Laplacian is the application of the divergence operator to the result of the gradient 
operator. 

The continuity equation states that the divergence of the 
velocity field is 0, in other words the velocity field is divergence-free. This 
means that for every point in the field, the flow in is equal to the flow out
\cite{originalSnowThesis}. 

The momentum equation consists of four terms;
\begin{description}
	\item[Advection,] the term $-(u \cdot \nabla)u$ describes how the fluid's 
	velocity causes it's density and other properties to move with it's flow. 
	The motions in a fluid transports itself along the flow. This is called 
	advection. 
	\item[Pressure,] the term $-\frac{1}{\rho}\nabla p$ accounts for the pressure 
	in the fluid and is a result of the molecules in the fluid pushing on each 
	other. This leads to changes in the fluid's velocity. 
	\item[Diffusion,] the term $\nu \nabla^2 u$ describes how the thickness of the 
	fluid affects the change in velocity, a fluids thickness is also called viscosity. 
	Viscosity describes how resistant the fluid is to flow. 
	\item[External forces,] the term $f$ accounts for external forces, like gravity.
\end{description}

To model fluid flow realistically a method for solving Navier-Stokes equations is 
required. However solving the equations is difficult because of their nonlinearity. 
There are no general analytic solution to them, it is therefore necessary to make 
some assumptions about the system the equations are applied to. For example for 
visualization purposes the appearance of the fluid is more important than accuracy 
\cite{smokeAndFire}. 

\subsection{Air}

In the fluid models for air, some of the terms in the momentum equation are left 
out. If we assume that the air has zero viscosity and that the density is equal 
to one, then the incompressible Navier-Stokes equations are reduced to the incompressible 
Euler equations. We also assume that the force of gravity on air is negligible 
compared to the advection and pressure terms, the term for external forces can 
therefore be omitted\cite{originalSnowThesis}. 

\begin{equation} 
	\tag{momentum equation}
	\pOne{t}u  = -(u \cdot \nabla)u - \nabla p
\end{equation}

\begin{equation}
	\tag{continuity equation}
	\nabla \cdot u = 0
\end{equation}
