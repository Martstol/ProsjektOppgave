\section{Computational Fluid Dynamics}

Fluid flows can be described by partial differential equations for the conservation 
of mass, momentum and energy. In computational fluid dynamics (CFD) the partial 
differential equations are replaced by a system of linear equations that can be 
solved by a computer. 

\subsection{Navier-Stokes Equations}

Navier-Strokes equations are a set of partial differential equations which describes 
fluid flow\cite{fluidDynamicsIntro}. By assuming the fluid is incompressible, 
the equations are simplified to the Navier-Strokes equations for incompressible 
flow\cite{originalSnowThesis}. 

\begin{equation} 
	\tag{momentum equation}
	\pOne{t}u  = -(u \cdot \nabla)u - \frac{1}{\rho}\nabla p + \nu \nabla^2 u + f
\end{equation}

\begin{equation}
	\tag{continuity equation}
	\nabla u = 0
\end{equation}

Here $u$ is the velocity vector field, $p$ is the pressure field, $\rho$ is density,
$\nu$ is the kinematic viscosity and $f$ are external forces working on the fluid
\cite{gpuSnowThesis}. 

The continuity equation states that the divergence of the 
velocity field is 0, in other words the velocity field is divergence-free. This 
means that for every point on the field, the flow in is equal to the flow out
\cite{originalSnowThesis}. 

The momentum equation consists of four terms;
\begin{description}
	\item[Advection,] 
	\item[Pressure,]
	\item[Diffusion,]
	\item[External forces,] the term $f$ accounts for external forces, like gravity.
\end{description}

"The problem of simulating realistic fluids is equivalent
to writing good solvers for the Navier-Stokes
equations. Solving them is central to research in areas
as diverse as aeronautics, combustion science, biophysics,
and mechanical engineering. However, they
are also notoriously difficult to solve, due to their
nonlinearity, complicating the relationship between
causes and effects."

In the fluid models for air, some of these terms are left out. If we assume that 
the air has zero viscosity and that the density is equal to one, then the 
incompressible Navier-Stokes equations reduce to the incompressible Euler 
equations. We also assume that the force of gravity on air is negligible compared 
to the advection and pressure terms, the term for external forces can therefore be 
emitted. 

\begin{equation} 
	\tag{momentum equation}
	\pOne{t}u  = -(u \cdot \nabla)u - \nabla p
\end{equation}

\begin{equation}
	\tag{continuity equation}
	\nabla u = 0
\end{equation}
