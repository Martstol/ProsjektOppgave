\section*{Linear Solvers}

A system of linear equations with $N$ unknowns can be written as 

$Ax = b$

where $A$ is a $N \times N$ matrix, and $x$ and $b$ are column vectors with $N$ 
elements each. We want to solve the system and find an expression for $x$ by calculating
$\inv{A}$. The solution is written as 

$x = \inv{A}b$

\subsection*{Direct Methods}

Direct methods work by directly solving the system of linear equations. In other 
words, they compute $\inv{A}$. There are different methods for finding the inverse
of $A$, two of the most well known are Gaussian elimination, also known as Gauss-
Jordan elimination, and LU decomposition.

\subsubsection*{Gaussian elimination}

Gaussian elimination solves the system by performing row operations on the matrix 
$A$, to turn $A$ into the identity matrix. If these row operations are also performed 
on the vector $b$ then $b$ will be the solution vector when the Gaussian elimination 
is complete.

This method does not require you directly find $\inv{A}$, but finds the inverse 
indirectly. The combination of performing all the row operations is the same as 
multiplying with $\inv{A}$.

Gaussian elimination has O($n^3$) computational complexity, which means using 
Gaussian elimination is not feasible for solving equations with more than a few 
thousand unknowns with today's computing power.

\subsubsection*{LU decomposition}

\subsection*{Iterative Methods}

Describe iterative methods

\subsubsection*{Jacobi Method}

Describe the Jacobi method

\subsubsection*{Gauss-Seidel Method}

Describe Gauss-Seidel method

\subsubsection*{Successive Over-relaxation}

Describe successive over-relaxation
