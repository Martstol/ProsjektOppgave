\section{Linear Solvers}

A system of linear equations with $N$ unknowns can be written as 

$$Ax = b$$

where $A$ is a $N \times N$ matrix, and $x$ and $b$ are column vectors with $N$ 
elements each. $A$ and $b$ are known before hand, and we want to solve for $x$.

This can either be done with direct methods which compute the solution in a finite 
number of steps or by using an iterative method which iteratively improves an initial 
guess until we reach convergence. 

\subsection{Direct Methods}

Direct methods are methods that solve the system of linear equations in a finite
number of steps. Two well known methods are Gaussian elimination, and LU factorization.

\subsubsection{Gaussian elimination}

Gaussian elimination solves the system by performing row operations on the matrix 
$A$, to turn $A$ into the identity matrix. If these row operations are also performed 
on the vector $b$ then $b$ will be the solution vector when the Gaussian elimination 
is complete.

Gaussian elimination has O($n^3$) computational complexity, which means using 
Gaussian elimination is not feasible for solving equations with more than a few 
thousand unknowns with today's computing power.

\subsubsection{LU factorization}

The LU in LU factorization stands for "Lower Upper". It works by factoring the 
matrix into the product of a lower and an upper triangular matrix. 

$$A = LU$$

To solve the linear system 

$$LUx = b$$

We first solve 

$$Ly = b$$

for $y$ by forward substitution. Then we solve 

$$Ux = y$$

for $x$ by backward substitution.

LU factorization also has computational complexity O($n^3$), however the factorization part of 
LU factorization requires half the number of floating point operations compared to Gaussian elimination, 
and forward and backward substitution is O($n^2$)\cite{Kreyszig}.

\subsection{Iterative Methods}

Iterative methods start from an initial guess as to what the correct solution is, 
with the zero vector being a common choice for the initial solution. Then this 
solution is improved by repeating a procedure multiple times, with each step 
giving an improved approximation to the solution of the problem. Iterative methods 
are not guaranteed to produce a correct solution in a given amount of steps. 

Iterative methods stop when a stopping criteria is met, for example when the difference 
between the right-hand side and the left-hand side is small or the change in the approximate 
solution is below a tolerance level given by the user. 

\subsubsection{Jacobi Method}

The Jacobi method works by splitting the matrix $A$ into a diagonal matrix $D$ 
and the lower and upper triangular matrices $L$ and $U$. 

$$A = L+D+U$$

The matrix $D$ is diagonal and contains all the elements on $A$'s diagonal. $L$ 
is lower triangular and contains all the elements from $A$ below the diagonal. 
$U$ is upper triangular and contains all the elements from $A$ above the diagonal.

$$(L+D+U)x = b$$
$$Dx = (b - (L+U)x)$$
$$x = \inv{D}(b - (L+U)x)$$

The system is then solved iteratively by computing 

$$x^{(k+1)} = \inv{D}(b - (L+U)x^{(k)})$$

Written in element form

$$ x_i^{(k+1)} = \frac{1}{a_{ii}} \Big( b_i - \sum_{j \neq i} a_{ij} x_j^{(k)} \Big), ~ i = 1, 2, \ldots, n $$

Because the new value of each $x_i$ is independent of the new value for all the 
other elements in the $x$ vector, the Jacobi method is trivially parallel.

This method converges for every initial guess, if and only if the spectral
radius of $I - A$ is less than 1\cite{Kreyszig}.

\subsubsection{Gauss-Seidel Method}

Gauss-Seidel works similarly to the Jacobi method, by splitting $A$ into $D$, 
$L$ and $U$. The main difference is how the equation is rearranged and the consequences 
when parallelizing the iterative solver. 

$$ (L+D)x = b - Ux $$
$$ (L+D)x^{(k+1)} = b - Ux^{(k)} $$
$$ Dx^{(k+1)} = b - Lx^{(k+1)} - Ux^{(k)} $$
$$ x^{(k+1)} = \inv{D}(b - Lx^{(k+1)} - Ux^{(k)}) $$

The value of $x^{(k+1)}$ has to be computed sequentially using forward substitution. 
There is no room for parallelization. Written in element-based form:

$$ x_i^{(k+1)} = \frac{1}{a_{ii}} \Big( b_i - \sum_{j < i} a_{ij} x_j^{(k+1)} 
- \sum_{j > i} a_{ij} x_j^{(k)} \Big), ~ i = 1, 2, \ldots, n $$

As we see from the first sum, the new value of $x_i$ depends on the new value of 
all previous elements in the $x$ vector.

It is possible to create a parallel implementation of Gauss-Seidel by changing
the order new values of $x$ is calculated. This is known as Red-Black
ordering. The new ordering depends on the linear system.

The Gauss-Seidel method converges if $||C|| < 1$, where $C = -\inv{(I+L)}U$ and 
$||C||$ is some matrix norm of $C$\cite{Kreyszig}.

\subsubsection{Successive Overrelaxation}

Successive overrelaxation (SOR) is a modified version of Gauss-Seidel with the
goal of having faster convergence\cite{Kreyszig}. This method differs from Gauss-Seidel by
introducing a constant $\omega > 1$, which is called the \emph{overrelaxation
factor}. To get the equation for SOR we first add and subtract $x^{(k)}$ on the right 
hand side of Gauss-Seidel.

$$ Dx^{(k+1)} = x^{(k)} + b - Lx^{(k+1)} - (U + I)x^{(k)} $$

Then we introduce the overrelaxation factor $\omega > 1$ to get the SOR formula 
for Gauss-Seidel.

$$ Dx^{(k+1)} = x^{(k)} + \omega \big( b - Lx^{(k+1)} - (U + I)x^{(k)} \big) $$

A recommended value for the overrelaxation factor is $\omega = \frac{2}{1 +
\sqrt{1 - \rho}} $, where $\rho$ is the spectral radius of $-\inv{(I + L)}U$\cite{Kreyszig}. 

\subsection{Krylov subspace methods}

TODO write stuff
