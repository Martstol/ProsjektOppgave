\section*{Linear Solvers}

A system of linear equations with $N$ unknowns can be written as 

$Ax = b$

where $A$ is a $N \times N$ matrix, and $x$ and $b$ are column vectors with $N$ 
elements each. We want to solve the system and find an expression for $x$ by calculating
$\inv{A}$. The solution is written as 

$x = \inv{A}b$

\subsection*{Direct Methods}

Direct methods work by directly solving the system of linear equations. In other 
words, they compute $\inv{A}$. There are different methods for finding the inverse
of $A$, two of the most well known are Gaussian elimination, also known as Gauss-
Jordan elimination, and LU decomposition, also known as LU factorization.

\subsubsection*{Gaussian elimination}

Gaussian elimination solves the system by performing row operations on the matrix 
$A$, to turn $A$ into the identity matrix. If these row operations are also performed 
on the vector $b$ then $b$ will be the solution vector when the Gaussian elimination 
is complete.

This method does not require you directly find $\inv{A}$, but finds the inverse 
indirectly. The combination of performing all the row operations is the same as 
multiplying with $\inv{A}$.

Gaussian elimination has O($n^3$) computational complexity, which means using 
Gaussian elimination is not feasible for solving equations with more than a few 
thousand unknowns with today's computing power.

Another problem with Gaussian elimination is that it is numerically unstable. 
When eliminating non-zero elements the algorithm may have to divide all the elements 
in a row with the value of the first coefficient on that row. If this number is 
close to zero any rounding error would be amplified.

\subsubsection*{LU decomposition}

The LU in LU decomposition stands for "Lower Upper". It works by factoring the 
matrix into the product of a lower and an upper triangular matrix. 

$A = LU$

To solve the linear system 

$LUx = b$

We first solve 

$Ly = b$

for $y$ by forward substitution. Then we solve 

$Ux = y$

for $x$ by backward substitution.

LU decomposition also has computational complexity O($n^3$). However LU decomposition 
is numerically stable, making it more suitable for a computer.

\subsection*{Iterative Methods}

Describe iterative methods

\subsubsection*{Jacobi Method}

Describe the Jacobi method

\subsubsection*{Gauss-Seidel Method}

Describe Gauss-Seidel method

\subsubsection*{Successive Over-relaxation}

Describe successive over-relaxation
