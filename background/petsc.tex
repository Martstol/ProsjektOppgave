\section{PETSc}

PETSc is short for Portable, Extensible Toolkit for Scientific Computation. 
It contains various data structures and functions for 
solving  in parallel. PETSc supports distributed 
memory systems with MPI, shared memory systems using Pthreads or OpenMP as well 
as GPGPU computing using CUDA or OpenCL. PETSc is open source and distributed 
under the 2-clause BSD license\cite{petsc-web-page}.

\subsection{Programming Model}
PETSc is primarily written in C, but depends on certain Fortran libraries like BLAS 
and LAPACK. PETSc is written with an object-oriented design pattern.
There are object types for representing both data structures and solvers.
Instead of accessing data directly, all manipulation of data structures uses functions
that abstracts away the underlying implementation, there is no direct data access. 
PETSc' interface is designed based on the operations you perform on the data, rather 
than the data itself. This hides complications as vectors being distributed to 
several machines running in parallel.

\subsection{Solvers}
PETSc implements a large number of parallel numerical solvers, both for linear and 
nonlinear equations and ordinary differential equation solvers. For linear systems, 
PETSc implements both direct and iterative methods. The iterative methods are of the 
type called Krylov subspace methods. 
