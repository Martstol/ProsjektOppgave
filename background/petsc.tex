\section{PETSc}

PETSc is short for Portable, Extensible Toolkit for Scientific Computation. 
It contains various data structures and functions for 
high performance parallel computations. PETSc supports distributed 
memory systems with MPI, shared memory systems using Pthreads or OpenMP as well 
as GPGPU computing using CUDA\footnote{\url{https://developer.nvidia.com/cuda-toolkit}} 
or OpenCL\footnote{\url{https://www.khronos.org/opencl/}}. PETSc is open source and distributed 
under the 2-clause BSD license\cite{petsc-web-page}.

\subsection{Programming Model}

PETSc is primarily written in C, but depends on certain Fortran libraries like 
BLAS\footnote{\url{http://www.netlib.org/blas/}} and LAPACK
\footnote{\url{http://www.netlib.org/lapack/}}. PETSc is written with an object-
oriented design pattern. There are object types for representing both data structures 
and solvers. Instead of accessing data directly, all manipulation of data structures 
uses functions that abstracts away the underlying implementation, there is no direct 
data access. PETSc' interface is designed based on the operations you perform on 
the data, rather than the data itself. This hides complications as vectors being 
distributed to several machines running in parallel.

\subsection{Solvers}

PETSc implements a large number of parallel numerical solvers, both for linear and 
nonlinear equations and ordinary differential equation solvers. For linear systems, 
PETSc implements both direct and iterative methods. The iterative methods implemented 
in PETSc are Krylov subspace methods, discussed in the section for linear solvers.

\subsection{GPGPU}

GPGPU stands for general purpose computing on graphics processing units. It was 
first implemented as a part of PETSc in 2010\cite{minden2010preliminary} by Victor 
Minden et al. PETSc can utilize CUDA through Thrust\footnote{\url{http://thrust.github.io/}} 
and Cusp\footnote{http://cusplibrary.github.io/} and OpenCL with ViennaCL
\footnote{\url{http://viennacl.sourceforge.net/}}. PETSc's installation instructions 
points out that OpenCL is less a burden on the build system than CUDA, and makes 
installation easier as it works with any host compiler. 

The changes required to have PETSc utilize the GPU is minimal. For the GPU 
implementation PETSc adds new subclasses of the vector and matrix classes that 
stores the data on both the GPU and the CPU. Krylov solvers utilize GPU implementations 
of the functions defined for these classes.

To utilize the GPU the application programmer have to specify a GPU vector type 
for vectors and a GPU matrix type for matrices. This can either be done by using 
the $VecSetType$ and $MatSetType$ functions or by using the $VecSetFromOptions$ 
and $MatSetOptions$ functions and specifying the type by adding $-vec\_type$ and 
$-mat\_type$ along with a GPU type on the commandline during runtime. 

All Krylov subspace methods implemented in PETSc function on the GPU and require 
no copies between the CPU and GPU during a Krylov solve. With the exception 
of KSPIBCGS\footnote{Improved Stabilized version of BiConjugate Gradient Squared},
as this method requires direct access to the underlying data stored in the vectors 
and would require a complete rewrite.
