This chapter will describe the theoretical foundation material of the project. 
The first section of this chapter deals with computational fluid dynamics (CFD) 
for the wind simulation as air can be modeled approximately as a fluid with 
zero viscosity and a density equal to one\cite{originalSnowThesis}.

The second section deals with partial differential equations and boundary value
problems. This section will describe the Poisson equation which is solved as a 
part of the wind simulation.

The third section introduces the finite difference method for creating systems of
linear equations approximating partial differential equations and discusses
the discretized Poisson equation.

The fourth section deals with the topic of linear solvers, explaining both direct
methods and iterative methods. This section describes two direct
methods for solving systems of linear equations, Gaussian elimination and LU
decomposition and three iterative methods, the Jacobi method, Gauss-Seidel and
successive overrelaxation. This section also discuss a class of iterative methods 
known as Krylov subspace methods. 

The fifth and final section will describes the Portable, Extensible Toolkit for
Scientific Computation, PETSc\footnote{\url{http://www.mcs.anl.gov/petsc/}}, a C 
library that was used in the implementation.
