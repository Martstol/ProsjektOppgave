\chapter{Introduction}

This project will continue work on the HPC-lab's snow simulator by writing an
implementation of the wind simulator with the library PETSc and comparing the
performance of the PETSc implementation with the previously implemented wind
simulator. The snow simulator simulates falling snow particles and how they are
affected by the wind and terrain. Realistic modeling of how the wind is affected
by terrain is a challenging topic and an important part of simulating natural
phenomenons like the weather. The snow simulator uses computational fluid dynamics
to model the behavior of the wind and how it is affected by the terrain.
In computational fluid dynamics the behavior of a fluid is modeled by a set of
partial differential equations which can be solved to get the fluids velocity.

%In this project a library for scientific computation called PETSc will be used
%to implement the wind simulation.

%General purpose computing on graphics processing units (GPGPU) is a recently
%developed field of high performance computing that use graphics processing units
%(GPU) to accelerate the computation of highly parallel problems. While GPUs was
%originally developed for rasterisation of three dimensional vector graphics, the
%parallel design of GPUs has lead to research into using them for computationally
%intensive tasks. This was at first done by writing shading programs that ran on
%the programmable graphics pipeline and in 2006 NVIDIA introduced CUDA, a platform
%for parallel computations on NVIDIA GPUs. Later in 2008 Apple released OpenCL as
%an open standard for writing parallel programs for GPUs, CPUs and other types of
%processors in collaboration with AMD, NVIDIA, IBM, Intel and Qualcomm. The Khronos
%Group is responsible for maintaining OpenCL. While CUDA and OpenCL are notable
%improvements from using the programmable graphics pipeline for GPGPU programming,
%they are still low level APIs that are challenging to program for and may lead to
%complex software design. 

\section{Problem Description}

This project will implement a wind simulator in the HPC-lab's snow simulator 
using the PETSc library. PETSc was previously implemented in the CPU version
of the snow simulator by Mads Buvik Sandvei for his specialization project.
This project will implementing PETSc in the GPU version of the snow simulator
and utilize PETSc's GPU solvers in the wind simulation. 


\section{Outline}

Chapter 2 covers the theoretical foundation material required for the wind
simulation. The chapter looks at the computational fluid dynamics model that is
used to simulate the wind and the assumptions made for the implementation. This
chapter also covers the finite difference method and the topic of linear
solvers. Chapter 3 introduces the current version of the HPC-lab's snow
simulator. The history of the snow simulator is covered and the structure of the
code base is briefly outlined. Chapter 4 derives the discretized mathematical
expressions that have been implemented in the snow simulator. Chapter 5 covers
the details of the implementations that have been done for this project. Chapter
6 presents the results of the implemented wind simulation. Chapter 7 discusses
the results and chapter 8 presents the conclusions. Finally in chapter 9
possible future work to improve the implementation is presented.
