\section{PETSc}

This section describes which PETSc functions were used and how they were used in 
the implementation of the linear solver\cite{petsc-user-ref}. 

\subsection{Setup}

The $DM$ object and $MatSetValuesStencil$ function are the primary functionality from 
PETSc used when creating the matrix $A$\cite{petsc-user-ref}. DM is short for 
Data Management and is used to manage the mapping between the distributed PETSc 
algebraic structures (Vec and Mat) and the structure of the mesh grids for the PDE. 
$MatSetValuesStencil$ handles the mapping from the physical coordinates of the 
internal nodes in the PDE matrix to their global ordering. 

The $DM$ object is used to get the process local domain of the matrix, 
making it possible to set the values of the matrix when it is distributed across 
different machines in a distributed memory system. $MatSetValuesStencil$ is used 
to set the values of the matrix.

To assemble the right hand side vector, $b$ the To create the mapping the function 
$DMDACreate3D$ was used.

\subsection{Solver}

The linear solvers used in PETSc are called KSP, which stands for Krylov subspace 
methods. 
