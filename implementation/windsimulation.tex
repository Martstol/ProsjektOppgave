\section{Wind Simulation}

This section will describe how the formulas derived in chapter \ref{chap:windsim}
for the wind simulation was implemented and how PETSc was used in the
implementation. The wind simulation consists of three steps, advection, solving
the Poisson equation and projection. 

The wind simulator uses a number of data structures for storing the data used
during simulation. Most data structures represent a three dimensional volume.
The following are important data structures used by the wind simulation.
\begin{description}
	\item[A,] a PETSc matrix storing the $A$ matrix for the discretized Poisson
		equation.
	\item[b,] a PETSc vector storing the right-hand side of the discretized
		Poisson equation.
	\item[p,] a PETSc vector storing the solution to solving the discretized
		Poisson equation. This vector represents a three dimensional volume
		storing the pressure field for the wind simulation.
	\item[velocity fields,] the velocity field represents a three dimensional
		volume storing the wind velocity. There are three instances of the velocity
		field. The velocity field from the previous iteration, the intermediate
		velocity field calculated by the advection step and the new velocity
		field calculated during this iteration.
	\item[obstacles,] the obstacles vector represents a three dimensional volume
		of integers where the bits are set depending on how the terrain is
		formed. The first bit is set to 1 if the cell is blocked by the terrain,
		while the next 26 bits stores obstacle information about the 26
		neighboring cells.
\end{description}

\subsection{Advection}

\begin{figure}[ht]
	\center
	\includegraphics[width=0.6\textwidth]{images/trilinear_interpolation}
	\caption{Illustration of trilinear interpolation. The value at the point
	$C$ is linearly interpolated from the 8 corner points of the cube.}
	\label{fig:trilinearinterpolation}
\end{figure}

\subsection{Solve Poisson Equation}

The Poisson equation is solved using the KSP object from PETSc. KSP stands for
Krylov subspace methods, discussed in the background chapter.

\subsubsection{Creating the Linear System}

The linear system $Ax = b$ consists of the matrix $A$, and the vectors $x$ and
$b$. The vector $x$ is set to store the initial guess for the linear solver,
which was chosen to be null vector, $x = 0$. The vector $b$ is the divergence of
the result of the advected velocity field, $b = \nabla \cdot u^*$. How to
calculate the values for $b$ is shown in chapter \ref{chap:windsim}, in the
section on the Poisson equation. 

\subsubsection{Solver Setup}

\subsection{Projection}

