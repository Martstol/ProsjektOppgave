\section{C++ Wrappers}

PETSc is written in the C programming language and features an object oriented
design pattern, with all the disadvantages and advantages that entails. In order
to write maintainable, high level code that follows C++ idioms, C++ wrapper
classes have been created for the PETSc classes that was used in the
implementation.

The C++ wrappers serve the primary purpose of simplifying coding with PETSc in
C++. This is achieved by utilizing namespaces, classes, RAII and exceptions.

\subsection{Namespace}

Namespace is a tool in C++ for avoiding name collisions.

All C++ wrapper functions and classes have been put in a separate namespace
called $petscpp$, for PETSc plus plus.

\subsection{RAII}

RAII is short for Resource Acquisition Is Initialization and is a programming
idiom used in C++. In RAII holding resources is tied to an object's lifetime,
beginning at an object's creation and ending when the object is either manually
destroyed or goes out of scope.

\subsubsection{Constructor and Destructor}



\subsubsection{Copy and Move}



\subsection{Exceptions}

All PETSc functions return and integer error code of the type PetscErrorCode.
This is a typical method for handling errors in C as C does not have direct
support for error handling. C++ on the other hand does offer error handling in
the form of exceptions which has been used to handle errors in the PETSc C++
wrapper library. If an error code different from 0 is returned, an exception
is thrown with the 

In practice many functions in PETSc always returns the value 0, which means that
no error occurred. 
In these cases the extra code made necessary by error checking is not needed.
Because error handling is done internally in the C++ class functions, the
programmer does not have to check if the PETSc function returns an error code
every time the function is used.
