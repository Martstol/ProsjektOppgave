\chapter{Background}

Topics in this chapter:\\
Note: Question marks indicate chapters that I aren't sure if I should / have the time to include
\begin{itemize}
	\item PDE
	\begin{itemize}
		\item Computational Fluid Dynamics
		\item Poisson Equation
	\end{itemize}
	\item Direct solvers(?)
	\begin{itemize}
		\item Fourier transform(?)
	\end{itemize}
	\item Numerical approximations to PDEs
	\begin{itemize}	
		\item Finite difference method
	\end{itemize}
	\item Linear Solvers
	\begin{itemize}
		\item Direct methods
		\item Iterative methods
		\begin{itemize}
			\item Jacobi method
			\item Gauss-Seidel method
			\item SOR
		\end{itemize}
	\end{itemize}
	\item Petsc
	\begin{itemize}
		\item MPI
		\item Blas, lapack
	\end{itemize}
\end{itemize}

\section*{Partial Differential Equations}

A Partial Differential Equation (PDE) is a differential equation with multiple 
unknown variables and their partial derivatives. PDEs can be used to model physical 
processes in multiple dimensions. 

\subsection*{Poisson Equation}

In general, the Poisson Equation is defined as 

$\Delta \varphi = f$

where $\Delta$ is the Laplace operator and $\varphi$ and $f$ are real or complex 
functions. In three-dimensional Cartesian coordinates, the Poisson Equation can 
be written as 

$(\pTwo{x} + \pTwo{y} + \pTwo{z}) \varphi (x, y, z) = f(x, y, z)$


\section*{Finite Difference Method}

The Finite Difference Method (FDM) is a numerical method used to find an approximate 
solution to differential equations. It works by creating a discrete approximation 
of the derivative and using this approximation to write the differential equation 
as a system of linear equations.

There are different versions of the Finite Difference Method.
\begin{itemize}
	\item Forward difference, $\Delta_hf(x) = f(x+h) - f(x)$
	\item Backward difference, $\nabla_hf(x) = f(x) - f(x - h)$
	\item Central difference, $\delta_hf(x) = f(x + \frac{1}{2}h) - f(x - \frac{1}{2}h)$
\end{itemize}

It is easy to see the connection of the Forward difference and the definition of 
the derivative.

$f'(x) = \lim_{h \to 0} \frac{f(x+h) - f(x)}{h}$

We then give $h$ a fixed, non-zero, value instead of having $h$ approach 0. Therefore 
the forward difference divided by $h$ approximates the derivative when $h$ is small.

The second order derivative can be approximated using the following central differences formula:

$f''(x) \approx \frac{\delta_h^2f(x)}{h^2} = \frac{f(x+h) - 2f(x) + f(x-h)}{h^2}$

\subsection*{Discrete Poisson Equation}

Describe how the discrete Poisson Equation looks when using the second-order central difference approximation.

\section*{Linear Solvers}

Introduce the topic of Linear Solvers.

\subsection*{Direct Methods}

It would probably be a good idea to describe direct methods for solving linear equations to contrast with the iterative methods. \\
Briefly outline Gauss-Jordan elimination and LU-factorization. 

\subsection*{Iterative Methods}

Describe iterative methods

\subsubsection*{Jacobi Method}

Describe the Jacobi method

\subsubsection*{Gauss-Seidel Method}

Describe Gauss-Seidel method

\subsubsection*{Successive Over-relaxation}

Describe successive over-relaxation

\section*{Petsc}

Describe petsc.

\subsection*{BLAS and LAPACK}

Describe BLAS and LAPACK.

\subsection*{MPI}

Describe MPI. (Mentiod MIMD, SIMD etc?)


