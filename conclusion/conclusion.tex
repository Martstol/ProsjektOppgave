\chapter{Conclusion}

In this project PETSc has been used to implement a wind simulator for the HPC-
lab's snow simulator. This was done to investigate the performance gain
obtainable by using high level libraries to use the GPU for computationally
intensive problems instead of hand implementing solvers with low level APIs like
CUDA and OpenCL. A PETSc implementation also enables easy experimentation with
parallelization method as PETSc supports shared memory parallelization,
distributed memory parallelization and GPGPU. In addition PETSc implements a
large number of solvers making it easy to experiment with the choice of solver
for the wind simulator.

The results show that using the GPU implementation of PETSc gives large speedups
for the solver without having to make special adjustments to the code to account
for the GPU. However it does not achieve the same execution speed as hand tuned
implementations written in low level APIs like CUDA and OpenCL.

It is challenging to properly use the computational power of the GPU for
operations that are not straight forwards to define using functions implemented
in PETSc, for example trilinear interpolation required for the advection step of
the wind simulation. Because of this only the implementation of the Poisson
solver can use the GPU for computation. This severely limits the resolution of
the wind velocity field for the wind simulation.

Further more, configuring PETSc to use CUDA can be very challenging because of
issues with backwards compatibility. Using PETSc with CUDA could require running
the application on old GPU hardware, or updating PETSc itself to resolve the
backwards compatibility issues.
