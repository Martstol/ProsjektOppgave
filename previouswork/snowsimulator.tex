\section{Snow Simulator}

The snow simulator has been developed over several years by different students in 
the HPC-lab at NTNU. It simulates a finite number of snow particles and how they 
are affected by the wind and terrain. The following master theses and specialization 
projects have worked on the snow simulator:

\begin{itemize}
	\item Parallel Methods for Real-Time Visualization of Snow\cite{originalSnowThesis}
	\item Utilizing GPUs for Real-Time Visualization of Snow\cite{gpuSnowThesis}
	\item Enhancing and Porting the HPC-Lab Snow Simulator to OpenCL on Mobile Platforms
	\cite{openclSnowThesis}
	\item Terrain Rendering Techniques for the HPC-Lab Snow Simulator\cite{snowTerrainThesis}
	\item Enhancing the HPC-Lab Snow Simulator with More Realistic Terrains and Other Interactive Features
	\cite{realisticSnowTerrainThesis}
	\item Ray Tracing for Simulation of Wireless Networks in 3D Scenes\cite{rayTracingThesis}
	\item OpenACC-based Snow Simulation\cite{openAccThesis}
	\item Avalanche Simulations using Fracture Mechanics on the GPU\cite{avalancheThesis}
\end{itemize}

\subsection{Initialization}

\subsection{Main Loop}

\subsection{Wind Simulation}

The wind simulation calculates the three dimensional wind velocity field through
computational fluid dynamics (CFD). These calculations are done on the GPU using
CUDA. The discretization method being used is the finite difference method. The
boundary conditions for the CFD problem is decided by the terrain vertex data
and the external wind velocity at the borders of the domain. The external wind
velocity can be set by the user at runtime. The system is solved using a SOR-
solver and the boundary conditions are satisfied by adjusting cells that are
next to obstacles. Finally the wind velocity field is stored in texture memory.
Texture memory is used to allow efficient read-only access to the wind velocity
field by the snow particle simulator.
